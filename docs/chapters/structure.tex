\chapter{Structure of Electronic Transitions}
\label{c:structure_of_electronic_transitions}

\section{Total Energy}
\label{s:total_energy}

The total energy $E$ of a molecule is approximated by
\begin{equation*}
    E = E_e + E_v + E_r,
\end{equation*}
that is, the sum of the electronic, vibrational, and rotational energy terms. The same equation written in wavenumber units is
\begin{equation}
    T = T_e + G + F.
\end{equation}

\section{Vibrational Term}
\label{s:vibrational_term}

For the vibrational and rotational states of the molecule in different electronic states, the vibrating rotator model is used \cite{herzberg:diatomic}. The fourth-order approximation for the vibrational term $G(v)$ is
\begin{equation}
    G(v) = \omega_e\ab(v + \tfrac{1}{2}) - \omega_ex_e\ab(v + \tfrac{1}{2})^2 + \omega_ey_e\ab(v + \tfrac{1}{2})^3 + \omega_ez_e\ab(v + \tfrac{1}{2})^4 + \dotsb.
\end{equation}

\section{Rotational Term}
\label{s:rotational_term}

The rotational term $F_v(J)$ approximated to the third order is given as
\begin{equation}
    F_v(J) = B_vJ(J + 1) - D_vJ^2(J + 1)^2 + H_vJ^3(J + 1)^3 + \dotsb.
\end{equation}
The rotational constant $B_v$ can be approximated to the third order as (Herz. pp. 108)
\begin{equation*}
    B_v = B_e - \alpha_e\ab(v + \tfrac{1}{2}) + \gamma_e\ab(v + \tfrac{1}{2})^2 + \delta_e\ab(v + \tfrac{1}{2})^3 + \dotsb.
\end{equation*}
The centrifugal distortion constant $D_v$ can be approximated to the first order as (Herz. pp. 107)
\begin{equation}
    D_v = D_e + \beta_e\ab(v + \tfrac{1}{2}) + \dotsb.
\end{equation}
Finally, the third-order constant $H_v$ can be found as the first approximation (Herz. pp. 109)
\begin{equation}
    H_v \approx H_e.
\end{equation}

\section{Transition Wavenumbers}
\label{s:transition_wavenumbers}

The wavenumbers of the spectral lines corresponding to the transitions between two electronic states (in emission or absorption) are given by
\begin{equation}
    \nu = T' - T'' = (T_e' - T_e'') + (G' - G'') + (F' - F'').
\end{equation}
That is, the emitted or absorbed frequencies can be expressed as sums of their constituent parts:
\begin{equation*}
    \nu = \nu_e + \nu_v + \nu_r.
\end{equation*}
