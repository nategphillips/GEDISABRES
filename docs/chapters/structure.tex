\chapter{Structure of Electronic Transitions}
\label{c:structure_of_electronic_transitions}

\section{Total Energy}
\label{s:total_energy}

The total energy $E$ of a molecule is approximated by
\begin{equation*}
    E = E_{e} + E_{v} + E_{r},
\end{equation*}
that is, the sum of the electronic, vibrational, and rotational energy terms. The same equation written in wavenumber units is
\begin{equation}
    T = T_{e} + G + F.
\end{equation}

\section{Vibrational Term}
\label{s:vibrational_term}

For the vibrational and rotational states of the molecule in different electronic states, the vibrating rotator model is used \cite{herzberg:diatomic}. The fourth-order approximation for the vibrational term $G(v)$ is
\begin{equation}
    G(v) = \omega_{e}\ab(v + \tfrac{1}{2}) - \omega_{e}x_{e}\ab(v + \tfrac{1}{2})^{2} + \omega_{e}y_{e}\ab(v + \tfrac{1}{2})^{3} + \omega_{e}z_{e}\ab(v + \tfrac{1}{2})^{4} + \dotsb.
\end{equation}

\section{Rotational Term}
\label{s:rotational_term}

The rotational term $F_{v}(J)$ approximated to the third order is given as
\begin{equation}
    F_{v}(J) = B_{v}J(J + 1) - D_{v}J^{2}(J + 1)^{2} + H_{v}J^{3}(J + 1)^{3} + \dotsb.
\end{equation}
The rotational constant $B_{v}$ can be approximated to the third order as (Herz. pp. 108)
\begin{equation*}
    B_{v} = B_{e} - \alpha_{e}\ab(v + \tfrac{1}{2}) + \gamma_{e}\ab(v + \tfrac{1}{2})^{2} + \delta_{e}\ab(v + \tfrac{1}{2})^{3} + \dotsb.
\end{equation*}
The centrifugal distortion constant $D_{v}$ can be approximated to the first order as (Herz. pp. 107)
\begin{equation}
    D_{v} = D_{e} + \beta_{e}\ab(v + \tfrac{1}{2}) + \dotsb.
\end{equation}
Finally, the third-order constant $H_{v}$ can be found as the first approximation (Herz. pp. 109)
\begin{equation}
    H_{v} \approx H_{e}.
\end{equation}

\section{Transition Wavenumbers}
\label{s:transition_wavenumbers}

The wavenumbers of the spectral lines corresponding to the transitions between two electronic states (in emission or absorption) are given by
\begin{equation}
    \nu = T' - T'' = (T_{e}' - T_{e}'') + (G' - G'') + (F' - F'').
\end{equation}
That is, the emitted or absorbed frequencies can be expressed as sums of their constituent parts:
\begin{equation*}
    \nu = \nu_{e} + \nu_{v} + \nu_{r}.
\end{equation*}
