\chapter{Laser-Induced Fluorescence}

\section{Rate Equations}

\begin{figure}[H]
    \centering
    \begin{tikzpicture}
        \draw[thick] (6,0) -- (0,0) node[left] {$X\state{3}{\Sigma}_g^{-}(v'', J'') \quad N_1$};
        \draw[thick] (6,6) -- (0,6) node[left] {$B\state{3}{\Sigma}_u^{-}(v', J') \quad N_2$};

        \draw[->,ultra thick] (1,6) -- (1,0) node[midway,fill=base] {$W_e$};
        \draw[->,ultra thick] (1.5,6) -- (2,7) node[midway,fill=base] {$W_d$};
        \node at(2,7.2) {\ce{O2 -> 2O}};
        \draw[->,ultra thick] (2,0) -- (2,6) node[midway,fill=base] {$W_a$};
        \draw[->,ultra thick,decorate,decoration={snake,segment length=5mm}] (3,6) -- (3,0) node[midway,fill=base] {$W_{21}$};
        \draw[->,ultra thick] (4,6) -- (4,1) node[midway,fill=base] {$W_q$};
        \draw[thick] (3.5,1) -- (4.5,1) node[right] {$v \neq v''$};
        \draw[->,ultra thick,decorate,decoration={snake, segment length=5mm}] (5,6) -- (5,2) node[midway,fill=base] {$W_f$};
        \draw[thick] (4.5,2) -- (5.5,2) node[right] {$v \neq v''$};

        \draw[<-,ultra thick] (6.2,6.1) -- (7.8,6.4) node[midway,fill=base] {$R_{32}$};
        \draw[thick] (8,6.4) -- (10,6.4);
        \draw[thick] (8,6) -- (10,6) node[right] {$N_3$};
        \draw[thick] (8,5.6) -- (10,5.6);
        \draw[->,ultra thick] (6.2,5.9) -- (7.8,5.6) node[midway,fill=base] {$R_{23}$};

        \draw[->,ultra thick,decorate,decoration={snake,segment length=5mm}] (9,6) -- (9,2) node[midway,fill=base] {$W_f$};
        \draw[thick] (8.5,2) -- (9.5,2) node[right] {$v \neq v''$};

        \draw[<-,ultra thick] (6.2,0.1) -- (7.8,0.4) node[midway,fill=base] {$R_{41}$};
        \draw[thick] (8,0.4) -- (10,0.4);
        \draw[thick] (8,0) -- (10,0) node[right] {$N_4$};
        \draw[thick] (8,-0.4) -- (10,-0.4);
        \draw[->,ultra thick] (6.2,-0.1) -- (7.8,-0.4) node[midway,fill=base] {$R_{14}$};
    \end{tikzpicture}
    \caption{Four-level LIF model for the Schumann--Runge bands of molecular oxygen.}
\end{figure}

\subsection{Four-Level LIF}

The four-level equations are \cite[18]{grinsteadTemperatureMeasurementHighTemperature1995}
\begin{align*}
    \odv{N_1}{t} & = -(W_a + R_{14})N_1 + (W_e + W_{21})N_2 + R_{41}N_4                      \\
    \odv{N_2}{t} & = W_aN_1 - (W_f + W_d + W_q + W_e + W_{21} + R_{23})N_2 + R_{32}N_3 \\
    \odv{N_3}{t} & = R_{23}N_2 - (W_f + R_{32})N_3                                               \\
    \odv{N_4}{t} & = R_{14}N_1 - R_{41}N_4
\end{align*}
$W_a$ is the laser-stimulated absorption rate, $W_e$ the laser-stimulated emission rate, $W_{21}$ the spontaneous emission rate, $W_d$ the predissociation rate, $W_q$ the collisional quenching rate, $W_f$ the fluorescent radiative decay rate, and $R_{ij}$ the rotational energy transfer rates.

\subsection{Three-Level LIF}

The three-level equations are \cite[2]{diskin3LevelModelSchumann1996}
\begin{align*}
    \odv{N_1}{t} & = -W_aN_1 + (W_e + W_{21})N_2 + W_c\ab(\frac{f_b}{1 - f_b}N_3 - N_1) \\
    \odv{N_2}{t} & = W_aN_1 - (W_e + W_d + W_{21} + W_f + W_q)N_2                           \\
    \odv{N_3}{t} & = -W_c\ab(\frac{f_b}{1 - f_b}N_3 - N_1)
\end{align*}
where $f_b$ is the rotational Boltzmann fraction for the lower state.
Here, the laser rates are given by
\begin{align*}
    W_a & = I_l(t)B_{12}\phi(\nu) \\
    W_e & = I_l(t)B_{21}\phi(\nu),
\end{align*}
where $\phi(\nu)$ is the overlap integral between the transition and laser lineshapes, given as
\begin{equation*}
    \phi(\nu_t, \nu_l) = \int \phi_t(\nu)\phi_l(\nu) \odif{\nu}.
\end{equation*}
$I_l$ is the laser intensity, which can be modeled as a Gaussian distribution in time
\begin{equation*}
    I_l(t) = \frac{2\Phi}{\adif{t}}\sqrt{\frac{\ln{2}}{\pi}}\exp\ab[-4\ln{2}\ab(\frac{t - t_0}{\adif{t}})^2],
\end{equation*}
where $\Phi = E_l/A$ is the laser fluence.
