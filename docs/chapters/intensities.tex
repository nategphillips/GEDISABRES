\chapter{Intensities in Rotation-Vibration Spectra}
\label{c:intensities_in_rotation-vibration_spectra}

\section{Vibration}
\label{s:vibration}

\subsection{Partition Function}

The quantities given by
\begin{equation*}
    \eul^{-G_0(v)hc/kT}
\end{equation*}
give the relative numbers of molecules in the different vibrational levels relative to the number of molecules in the lowest vibrational level. A more useful equation gives the ratio between the number of molecules in a certain vibrational level relative to the total number of molecules as
\begin{equation*}
    \frac{N_v}{N} = \frac{\eul^{-G_0(v)hc/kT}}{Q_v}.
\end{equation*}
Here, $Q_v$ is the vibrational partition function and is given by
\begin{equation*}
    Q_v = \sum_{v=0}^{\infty}\eul^{-G_0(v)hc/kT} = 1 + \eul^{-G_0(1)hc/kT} + \eul^{-G_0(2)hc/kT} + \dotsb.
\end{equation*}

\subsection{Intensities}

\subsection{Franck-Condon Factors}

\section{Rotation}
\label{s:rotation}

\subsection{Partition Function}

The number of molecules $N_J$ in the rotational level $J$ of the lowest vibrational state at the temperature $T$ is proportional to
\begin{equation*}
    (2J + 1)\eul^{-F(J)hc/kT}.
\end{equation*}
For most practical cases like a rigid rotator with $\Lambda = 0$,
\begin{equation*}
    N_J \propto (2J + 1)\eul^{-BJ(J + 1)hc/kT}.
\end{equation*}
The number of molecules in the different rotational states does not increase linearly and goes through a maximum with the rotational quantum number
\begin{equation*}
    J\_{max} = \sqrt{\frac{kT}{2Bhc}} - \frac{1}{2}.
\end{equation*}

The ratio of the actual number of molecules in a given rotational state is given by
\begin{equation*}
    \frac{N_J}{N} = (2J + 1)\frac{\eul^{-F(J)hc/kT}}{Q_r},
\end{equation*}
where the rotational partition function $Q_r$ is given as
\begin{equation}
    Q_r = \sum_{J=0}^{\infty}(2J + 1)\eul^{-F(J)hc/kT} = 1 + 3\eul^{-F(1)hc/kT} + 5\eul^{-F(2)hc/kT} + \dotsb.
\end{equation}

For higher vibrational levels,
\begin{equation}
    N_J \propto (2J + 1)\eul^{-(G + F)hc/kT}.
\end{equation}
However, the factor $\eul^{-Ghc/kT}$ can be separated off since the distribution over the rotational levels is the same but the absolute population of all the levels is considerably smaller than for the lowest vibrational level.

\subsection{Intensities}

\subsection{H\"onl-London Factors}

\begin{table}[H]
    \centering
    \caption{H\"onl-London factors \cite{herzberg:diatomic}.}
    \label{t:honl-london_factors}
    \begin{tabular}{ccc}
        \toprule
        Branch  & Absorption                                                                & Emission                                                           \\
        \midrule
        \multicolumn{3}{c}{$\adif{\Lambda} = 0$ \textit{Transitions}}                                                                                            \\
        \cmidrule(lr){1-3}
        $S_J^R$ & $\dfrac{(J'' + 1 + \Lambda'')(J'' + 1 - \Lambda'')}{J'' + 1}$             & $\dfrac{(J' + \Lambda')(J' - \Lambda')}{J'}$                       \\
        \addlinespace[0.5em]
        $S_J^Q$ & $\dfrac{(2J'' + 1)\Lambda''^2}{J''(J'' + 1)}$                             & $\dfrac{(2J' + 1)\Lambda'^2}{J'(J' + 1)}$                          \\
        \addlinespace[0.5em]
        $S_J^P$ & $\dfrac{(J'' + \Lambda'')(J'' - \Lambda'')}{J''}$                         & $\dfrac{(J' + 1 + \Lambda')(J' + 1 - \Lambda')}{J' + 1}$           \\
        \addlinespace[0.5em]
        \multicolumn{3}{c}{$\adif{\Lambda} = +1$ \textit{Transitions}}                                                                                           \\
        \cmidrule(lr){1-3}
        $S_J^R$ & $\dfrac{(J'' + 2 + \Lambda'')(J'' + 1 + \Lambda'')}{4(J'' + 1)}$          & $\dfrac{(J' + \Lambda')(J' - 1 + \Lambda')}{4J'}$                  \\
        \addlinespace[0.5em]
        $S_J^Q$ & $\dfrac{(J'' + 1 + \Lambda'')(J'' - \Lambda'')(2J'' + 1)}{4J''(J'' + 1)}$ & $\dfrac{(J' + \Lambda')(J' + 1 - \Lambda')(2J' + 1)}{4J'(J' + 1)}$ \\
        \addlinespace[0.5em]
        $S_J^P$ & $\dfrac{(J'' - 1 - \Lambda'')(J'' - \Lambda'')}{4J''}$                    & $\dfrac{(J' + 1 - \Lambda')(J' + 2 - \Lambda')}{4(J' + 1)}$        \\
        \addlinespace[0.5em]
        \multicolumn{3}{c}{$\adif{\Lambda} = -1$ \textit{Transitions}}                                                                                           \\
        \cmidrule(lr){1-3}
        $S_J^R$ & $\dfrac{(J'' + 2 - \Lambda'')(J'' + 1 - \Lambda'')}{4(J'' + 1)}$          & $\dfrac{(J' - \Lambda')(J' - 1 - \Lambda')}{4J'}$                  \\
        \addlinespace[0.5em]
        $S_J^Q$ & $\dfrac{(J'' + 1 - \Lambda'')(J'' + \Lambda'')(2J'' + 1)}{4J''(J'' + 1)}$ & $\dfrac{(J' - \Lambda')(J' + 1 + \Lambda')(2J' + 1)}{4J'(J' + 1)}$ \\
        \addlinespace[0.5em]
        $S_J^P$ & $\dfrac{(J'' - 1 + \Lambda'')(J'' + \Lambda'')}{4J''}$                    & $\dfrac{(J' + 1 + \Lambda')(J' + 2 + \Lambda')}{4(J' + 1)}$        \\
        \bottomrule
    \end{tabular}
\end{table}

\section{Band Intensity}
\label{s:band_intensity}

The variation of the intensity of the lines in a rotation-vibration band as a function of $J$ is essentially given by the thermal distribution of the rotational levels. In this approximation, it is assumed that the transition probability is the same for all lines of a band. In reality, there is a slight dependence on $J$ and $\Delta{}J$. For the case of $\Lambda = 0$ when only the $P$ and $R$ branches appear, $J' + J'' + 1$ can be used in place of $2J + 1$; that is, the intensity depends on the mean value of $2J + 1$ for the upper and lower states. The $J$ value of the initial state should be used in the exponential term. For absorption, that is $J''$; for emission, $J'$.

The intensities of the lines of rotation or rotation-vibration bands in absorption are
\begin{equation}
    I\_{abs} = \frac{C\_{abs}\nu}{Q_r}(J' + J'' + 1)\eul^{-F(J'')hc/kT}.
\end{equation}
In emission, they are
\begin{equation}
    I\_{em} = \frac{C\_{em}\nu^4}{Q_r}(J' + J'' + 1)\eul^{-F(J')hc/kT}.
\end{equation}
