\chapter{Energy}

\section{Total Energy}

According to the Born--Oppenheimer approximation, the total energy of a molecule can be approximated as the sum of electronic, vibrational, and rotational components; that is \cite[149]{herzbergMolecularSpectraMolecular1950}
\begin{equation*}
    E = E_e + E_v + E_r.
\end{equation*}
In molecular spectroscopy, the components of interest are the electronic, vibrational, and rotational energies. Writing this equation using term values yields
\begin{equation*}
    T = T_e + G + F.
\end{equation*}
The wave numbers of the spectral lines corresponding to the transitions between two electronic states are given by
\begin{equation*}
    \nu = T' - T'' = (T_e' - T_e'') + (G' - G'') + (F' - F'').
\end{equation*}
This equation implies that the emitted or absorbed frequencies are composed of the electronic, vibrational, and rotational components:
\begin{equation*}
    \nu = \nu_e + \nu_v + \nu_r.
\end{equation*}

\section{Rotational Structure}

For a vibronic transition, the quantity
\begin{equation*}
    \nu_0 = \nu_e + \nu_v,
\end{equation*}
called the band origin, is constant, while $\nu_r$ is variable and depends on the value of the rotational quantum numbers of the upper and lower states \cite[168]{herzbergMolecularSpectraMolecular1950}. Therefore, the wavenumbers of a rovibrational band can be computed as
\begin{equation*}
    \nu = \nu_0 + F'(J') - F''(J'').
\end{equation*}

\section{Term Values}

The term values of the vibrating rotator can be expressed in the form of a double power series called the Dunham expansion, where \cite[109]{herzbergMolecularSpectraMolecular1950}
\begin{equation*}
    T = \sum_{lj}Y_{lj}\ab(v + \tfrac{1}{2})^l [J(J + 1)]^j.
\end{equation*}
The relations between Dunham coefficients $Y_{lj}$ and spectroscopic constants are shown below in \cref{t:dunham_coefficients}.
\begin{table}[H]
    \centering
    \caption{Relations between Dunham coefficients and spectroscopic constants \cite[419]{babouHighTemperatureNonequilibriumPartition2009}.}
    \label{t:dunham_coefficients}
    \begin{tabular}{c|ccccc}
        \toprule
        $l$ & $j = 0$            & $j = 1$           & $j = 2$      & $j = 3$ & $j = 4$ \\
        \midrule
        0   & $Y_{00}$           & $B_e$           & $-D_e$     & $H_e$ & $L_e$ \\
        1   & $\omega_e$       & $-\alpha_e$     & $-\beta_e$ &         &         \\
        2   & $-\omega_ex_e$ & $\gamma_e$      & $-g_e$     &         &         \\
        3   & $\omega_ey_e$  & $\delta_e$      & $-h_e$     &         &         \\
        4   & $\omega_ez_e$  & $\varepsilon_e$ & $-k_e$     &         &         \\
        5   & $\omega_ea_e$  & $\xi_e$         &              &         &         \\
        6   & $\omega_eb_e$  & $\eta_e$        &              &         &         \\
        7   & $\omega_ec_e$  & $\theta_e$      &              &         &         \\
        8   & $\omega_ed_e$  &                   &              &         &         \\
        9   & $\omega_ee_e$  &                   &              &         &         \\
        \bottomrule
    \end{tabular}
\end{table}
The first coefficient represents the addition to the zero-point energy above that of the anharmonic oscillator, which is \cite[109]{herzbergMolecularSpectraMolecular1950}
\begin{equation*}
    Y_{00} = \frac{B_e}{4} + \frac{\alpha_e\omega_e}{12B_e} + \frac{\alpha_e^2\omega_e^2}{144B_e^3} - \frac{\omega_ex_e}{4}.
\end{equation*}

The vibrational term value can be expanded as a power series following the approach of Dunham, which yields \cite[419]{babouHighTemperatureNonequilibriumPartition2009}
\begin{align*}
    G & = \omega_e\ab(v + \tfrac{1}{2}) - \omega_ex_e\ab(v + \tfrac{1}{2})^2 + \omega_ey_e\ab(v + \tfrac{1}{2})^3 + \omega_ez_e\ab(v + \tfrac{1}{2})^4 + \omega_ea_e\ab(v + \tfrac{1}{2})^5 \\
    & + \omega_eb_e\ab(v + \tfrac{1}{2})^6 + \omega_ec_e\ab(v + \tfrac{1}{2})^7 + \omega_ed_e\ab(v + \tfrac{1}{2})^8 + \omega_ee_e\ab(v + \tfrac{1}{2})^9 + \dotsb.
\end{align*}
Similarly, expansion of the rotational term value gives
\begin{equation*}
    F = B_vJ(J + 1) - D_v[J(J + 1)]^2 + H_v[J(J + 1)]^3 + L_v[J(J + 1)]^4 + \dotsb.
\end{equation*}
Note that the rotational expansion in particular is only a rough approximation and does not include the fine structure effects caused by spin-spin, spin-orbit, or spin-rotation coupling.
These additions are made depending on the specific molecule under consideration.
The rotational constants present within the rotational term value are expressed as
\begin{align*}
    B_v & = B_e - \alpha_e\ab(v + \tfrac{1}{2}) + \gamma_e\ab(v + \tfrac{1}{2})^2 + \delta_e\ab(v + \tfrac{1}{2})^3 + \varepsilon_e\ab(v + \tfrac{1}{2})^4 + \xi_e\ab(v + \tfrac{1}{2})^5 + \eta_e\ab(v + \tfrac{1}{2})^6 \\
    & + \theta_e\ab(v + \tfrac{1}{2})^7 + \dotsb,                                                                                                                                                                                         \\
    D_v & = D_e + \beta_e\ab(v + \tfrac{1}{2}) + g_e\ab(v + \tfrac{1}{2})^2 + h_e\ab(v + \tfrac{1}{2})^3 + k_e\ab(v + \tfrac{1}{2})^4 + \dotsb,                                                                                   \\
    H_v & = H_e + \dotsb,
\end{align*}
and
\begin{equation*}
    L_v = L_e + \dotsb.
\end{equation*}

The Hamiltonian for the $X\state{3}{\Sigma}_g^{-}$ ground state of \ce{O2} can be expressed as the sum of rotational, spin-spin, and spin-rotation interactions.
Namely, \cite[1394]{amiotMagneticDipole1Dg1981}
\begin{equation*}
    H = H_r + H_{ss} + H_{sr},
\end{equation*}
where
\begin{align*}
    H_r  & = B\vb{N}^2 - D\vb{N}^4,                    \\
    H_{ss} & = \tfrac{2}{3}\lambda(3S_z^2 - \vb{S}^2),
\end{align*}
and
\begin{equation*}
    H_{sr} = \gamma\vb{N}\vdot\vb{S}.
\end{equation*}
The spin-spin and spin-rotation coupling constants, $\lambda$ and $\gamma$ respectively, can be written as
\begin{align*}
    \lambda & = \lambda_0 + \lambda_1\vb{N}^2 \\
    \gamma  & = \gamma_0 + \gamma_1\vb{N}^2.
\end{align*}

In a Hund's case (b) basis, the matrix representation of the three Hamiltonians for a given $J$ is \cite[1394]{amiotMagneticDipole1Dg1981}
\begin{align*}
    H_r                                          & = B\bmx{
        J(J - 1)                                       & 0                                                        & 0                   \\
        0                                              & (J + 1)(J + 2)                                           & 0                   \\
        0                                              & 0                                                        & J(J + 1)
    }                                                                                                                               \\
    & - D\bmx{
        J^2(J - 1)^2                               & 0                                                        & 0                   \\
        0                                              & (J + 1)^2(J + 2)^2                                   & 0                   \\
        0                                              & 0                                                        & J^2(J + 1)^2
    }                                                                                                                               \\
    H_{ss}                                         & = \lambda_0\bmx{
        \frac{2}{3} - \frac{2J}{2J + 1}                & \frac{2\sqrt{J(J + 1)}}{2J + 1}                          & 0                   \\
        \frac{2\sqrt{J(J + 1)}}{2J + 1}                & \frac{2}{3} - \frac{2(J + 1)}{2J + 1}                    & 0                   \\
        0                                              & 0                                                        & \frac{2}{3}
    }                                                                                                                               \\
    & + \lambda_1\bmx{
        \ab(\frac{2}{3} - \frac{2J}{2J + 1})J(J - 1)   & \frac{2\sqrt{J(J + 1)}}{2J + 1}(J^2 + J + 1)           & 0                   \\
        \frac{2\sqrt{J(J + 1)}}{2J + 1}(J^2 + J + 1) & \ab(\frac{2}{3} - \frac{2(J + 1)}{2J + 1})(J + 1)(J + 2) & 0                   \\
        0                                              & 0                                                        & \frac{2}{3}J(J + 1)
    }                                                                                                                               \\
    H_{sr}                                         & = \gamma_0\bmx{
        J - 1                                          & 0                                                        & 0                   \\
        0                                              & -(J + 2)                                                 & 0                   \\
        0                                              & 0                                                        & -1
    }
    + \gamma_1\bmx{
        J(J - 1)^2                                   & 0                                                        & 0                   \\
        0                                              & -(J + 2)^2(J + 1)                                      & 0                   \\
        0                                              & 0                                                        & -J(J + 1)
    }.
\end{align*}

Using the same Hamiltonian, the combined matrix elements in a Hund's case (a) basis can also be written as \cite[3]{cheungMolecularSpectroscopicConstants1986}
\begin{align*}
    F_2      & = T + Bx - Dx^2 + \tfrac{2}{3}\lambda - \gamma + \tfrac{2}{3}\lambda_Dx - \gamma_Dx, \\
    F_1F_3 & = \bmx{
        H_{11}     & H_{12}                                                                                     \\ H_{21} & H_{22}
    },
\end{align*}
where
\begin{align*}
    H_{11} & = T + B(x + 2) - D(x^2 + 8x + 4) - \tfrac{4}{3}\lambda - 2\gamma - \tfrac{4}{3}\lambda_D(x + 2) - 4\gamma_D(x + 1) \\
    H_{12} & = -2\sqrt{x}\ab[B - 2D(x + 1) - \tfrac{\gamma}{2} - \tfrac{2}{3}\lambda_D - \tfrac{1}{2}\gamma_D(x + 4)]             \\
    H_{21} & = H_{12}                                                                                                                 \\
    H_{22} & = T + Bx - D(x^2 + 4x) + \tfrac{2}{3}\lambda - \gamma + \tfrac{2}{3}x\lambda_D - 3x\gamma_D.
\end{align*}
Note that
\begin{equation*}
    x = J(J + 1).
\end{equation*}

\section{Main and Satellite Branches}

The six main branches in $\state{3}{\Sigma}\dash\state{3}{\Sigma}$ transitions are \cite[249]{herzbergMolecularSpectraMolecular1950}
\begin{align*}
    P_1 & = \nu_0 + F'_1(J - 1) - F''_1(J)  \\
    R_1 & = \nu_0 + F'_1(J + 1) - F''_1(J)  \\
    P_2 & = \nu_0 + F'_2(J - 1) - F''_2(J)  \\
    R_2 & = \nu_0 + F'_2(J + 1) - F''_2(J)  \\
    P_3 & = \nu_0 + F'_3(J - 1) - F''_3(J)  \\
    R_3 & = \nu_0 + F'_3(J + 1) - F''_3(J).
\end{align*}
The six satellite branches are
\begin{align*}
    \state{P}{Q}_{12} & = \nu_0 + F'_1(J - 1) - F''_2(J)  \\
    \state{R}{Q}_{21} & = \nu_0 + F'_2(J + 1) - F''_1(J)  \\
    \state{P}{Q}_{13} & = \nu_0 + F'_1(J - 1) - F''_3(J)  \\
    \state{R}{Q}_{31} & = \nu_0 + F'_3(J + 1) - F''_1(J)  \\
    \state{P}{Q}_{23} & = \nu_0 + F'_2(J - 1) - F''_3(J)  \\
    \state{R}{Q}_{32} & = \nu_0 + F'_3(J + 1) - F''_2(J).
\end{align*}
