\chapter{Hamiltonians}

\begin{table}[H]
    \centering
    \caption{Inherent angular momenta and their quantum numbers \cite[72]{lefebvre-brionSpectraDynamicsDiatomic2004}.}
    \begin{threeparttable}
        \begin{tabular}{lcccc}
            \toprule
            \multirow{2}{*}{\textbf{Type}}          & \multicolumn{2}{c}{\textbf{Total}} & \multicolumn{2}{c}{\textbf{Projection onto INA}\tnote{\dag}}                                                    \\
            \cmidrule(lr){2-3} \cmidrule(lr){4-5}
            & \textbf{Operator}                  & \textbf{QN}\tnote{\ddag}                             & \textbf{Operator} & \textbf{QN}                  \\
            \midrule
            Nuclear spin of A                       & $\vop{I}_A$                        & $I_A$                                   & $\sop{I}_{Az}$    & no standard                  \\
            Nuclear spin of B                       & $\vop{I}_B$                        & $I_B$                                   & $\sop{I}_{Bz}$    & no standard                  \\
            Orbital ang. mom. of the $i$th electron & $\vop{l}_i$                        & $l_i$                                   & $\sop{l}_{iz}$    & $\lambda_i = 0, 1, 2, \dots$ \\
            Spin of the $i$th electron              & $\vop{s}_i$                        & $s_i$                                   & $\sop{s}_{iz}$    & $\sigma_i = \pm\frac{1}{2}$  \\
            Rotational ang. mom.                    & $\vop{R}$                          & $R$                                     & $\sop{R}_z$       & zero                         \\
            \bottomrule
        \end{tabular}
        \begin{tablenotes}
        \item [\dag] internuclear axis
        \item [\ddag] quantum number
        \end{tablenotes}
    \end{threeparttable}
\end{table}

\begin{table}[H]
    \centering
    \caption{Coupled angular momenta and their quantum numbers \cite[72]{lefebvre-brionSpectraDynamicsDiatomic2004}.}
    \begin{tabular}{lcccc}
        \toprule
        \multirow{2}{*}{\textbf{Type}}        & \multicolumn{2}{c}{\textbf{Total}}                & \multicolumn{2}{c}{\textbf{Projection onto INA}}                                                         \\
        \cmidrule(lr){2-3} \cmidrule(lr){4-5}
        & \textbf{Operator}                                 & \textbf{QN}                             & \textbf{Operator} & \textbf{QN}                       \\
        \midrule
        Total nuclear spin                    & $\vop{I} = \vop{I}_A + \vop{I}_B$                 & $I$                                     & $\sop{I}_z$       & no standard                       \\
        Total ang. mom. of the $i$th electron & $\vop{j}_i = \vop{l}_i + \vop{s}_i$               & $j_i$                                   & $\sop{j}_i$       & $\omega_i = \lambda_i + \sigma_i$ \\
        Total electron orbital ang. mom.      & $\vop{L} = \sum_i\vop{l}_i$                       & $L$                                     & $\sop{L}_z$       & $\Lambda = 0, 1, 2, \dots$        \\
        Total electron spin ang. mom.         & $\vop{S} = \sum_s\vop{s}_i$                       & $S$                                     & $\sop{S}_z$       & $\Sigma$                          \\
        Total ang. mom. w/o any spin          & $\vop{N} = \vop{R} + \vop{L}$                     & $N$                                     & $\sop{N}_z$       & $\Lambda$                         \\
        Total ang. mom. w/o nuclear spin      & $\vop{J} = \vop{R} + \vop{L} + \vop{S}$           & $J$                                     & $\sop{J}_z$       & $\Omega = \Lambda + \Sigma$       \\
        Total ang. mom.                       & $\vop{F} = \vop{R} + \vop{L} + \vop{S} + \vop{I}$ & $F$                                     & $\sop{F}_z$       & no standard                       \\
        \bottomrule
    \end{tabular}
\end{table}

\begin{table}[H]
    \centering
    \caption{Hund's coupling cases and their respective descriptions \cite[2]{hougenCalculationRotationalEnergy1970} and \cite[103]{lefebvre-brionSpectraDynamicsDiatomic2004}.}
    \begin{tabular}{cccc}
        \toprule
        \textbf{Hund's Case} & \textbf{Rotational Energy} & \textbf{Good Quantum Numbers}                 & \textbf{Degeneracy}     \\
        \midrule
        (a)                  & $BJ(J + 1)$                & $\ket{JS\Omega\Lambda\Sigma}$                 & 2 or 1                  \\
        (b)                  & $BN(N + 1)$                & $\ket{JNS\Lambda(S_R)}$                       & $2(2S + 1)$ or $2S + 1$ \\
        (c)                  & $BJ(J + 1)$                & $\ket{J[J_a]\Omega}$                          & 2 or 1                  \\
        (d)                  & $BR(R + 1)$                & $\ket{JNS(S_R)J_+N_+S_+\Lambda_+l(l_R, s_R)}$ & $(2L + 1)(2S + 1)$      \\
        \bottomrule
    \end{tabular}
\end{table}

\section{Angular Momentum Commutation Rules}

\subsection{Space-Fixed Coordinates}

The commutation rules are \cite[73]{lefebvre-brionSpectraDynamicsDiatomic2004}
\begin{equation*}
    \comm{\sop{A}_I}{\sop{A}_J} = i\hbar\varepsilon_{IJK}\sop{A}_K.
\end{equation*}
From this, the matrix elements can be derived as
\begin{align*}
    \mel{A'\alpha'M_A'}{\vop{A}^2}{A\alpha M_A}   & = \hbar^2A(A + 1)\delta_{A'A}\delta_{\alpha'\alpha}\delta_{M_A'M_A}                                                                        \\
    \mel{A'\alpha'M_A'}{\sop{A}_Z}{A\alpha M_A}   & = \hbar M_A\delta_{A'A}\delta_{\alpha'\alpha}\delta_{M_A'M_A}                                                                              \\
    \mel{A'\alpha'M_A'}{\sop{A}^\pm}{A\alpha M_A} & = \hbar\exp[i(\theta_{M_A} - \theta_{M_A \pm 1})]\sqrt{A(A + 1) - M_A(M_A \pm 1)}\delta_{A'A}\delta_{\alpha'\alpha}\delta_{M_A'M_A \pm 1}.
\end{align*}

\subsection{Molecule-Fixed Coordinates}

\subsubsection{Normal Commutation Rules}

Normal commutation rules followed by the angular momentum operators $\vop{I}$, $\vop{L}$, $\vop{S}$, and $\vop{J}_a$ are \cite[74]{lefebvre-brionSpectraDynamicsDiatomic2004}
\begin{equation*}
    \comm{\sop{A}_i}{\sop{A}_j} = i\hbar\varepsilon_{ijk}\sop{A}_k.
\end{equation*}
For normal commutation, the molecule-fixed matrix elements for $\sop{A}_i$ are analogous to those for $\sop{A}_I$,
\begin{align*}
    \mel{A'\alpha'M_A'}{\vop{A}^2}{A\alpha M_A}   & = \hbar^2A(A + 1)\delta_{A'A}\delta_{\alpha'\alpha}\delta_{M_A'M_A}                                                                                \\
    \mel{A'\alpha'M_A'}{\sop{A}_z}{A\alpha M_A}   & = \hbar\alpha\delta_{A'A}\delta_{\alpha'\alpha}\delta_{M_A'M_A}                                                                                    \\
    \mel{A'\alpha'M_A'}{\sop{A}_\pm}{A\alpha M_A} & = \hbar\exp[i(\phi_{\alpha} - \phi_{\alpha \pm 1})]\sqrt{A(A + 1) - \alpha(\alpha \pm 1)}\delta_{A'A}\delta_{\alpha'\alpha \pm 1}\delta_{M_A'M_A}.
\end{align*}

\subsubsection{Anomalous Commutation Rules}

Anomalous commutation rules followed by the angular momentum operators $\vop{R}$, $\vop{F}$, $\vop{J}$, $\vop{N}$, and $\vop{O}$ are \cite[74]{lefebvre-brionSpectraDynamicsDiatomic2004}
\begin{equation*}
    \comm{\sop{A}_i}{\sop{A}_j} = -i\hbar\varepsilon_{ijk}\sop{A}_k.
\end{equation*}
For anomalously commuting operators, the diagonal matrix elements for $\vop{A}^2$ and $\sop{A}_z$ are identical to those for the normal case, but the off-diagonal elements are
\begin{equation*}
    \mel{A'\alpha'M_A'}{\sop{A}_\pm}{A\alpha M_A} = \hbar\exp[i(\phi_{\alpha} - \phi_{\alpha \mp 1})]\sqrt{A(A + 1) - \alpha(\alpha \mp 1)}\delta_{A'A}\delta_{\alpha'\alpha \mp 1}\delta_{M_A'M_A}.
\end{equation*}

\subsection{Useful Commutation Relations}

The following relations can be proven for any \textbf{normally commuting} angular momentum operator $\vop{A}$ using the commutation rules outlined above:
\begin{align*}
    \comm{\vop{A}^2}{\sop{A}_x}   & = \comm{\vop{A}^2}{\sop{A}_y} = \comm{\vop{A}^2}{\sop{A}_z} = 0 \\
    \comm{\sop{A}_z}{\sop{A}_\pm} & = \pm\hbar\sop{A}_\pm                                           \\
    \comm{\sop{A}_+}{\sop{A}_-}   & = 2\hbar\sop{A}_z.
\end{align*}

\section{Matrix Elements for Common Angular Momenta}

The matrix elements for the common angular momenta $\vop{J}$, $\vop{L}$, and $\vop{S}$ can be found as follows \cite[8]{hougenCalculationRotationalEnergy1970}.
Total angular momentum excluding nuclear spin $\vop{J}$:
\begin{align*}
    \mel{J\Omega}{\vop{J}^2}{J\Omega}         & = \hbar^2J(J + 1)                                                                        \\
    \mel{J\Omega}{\sop{J}_z}{J\Omega}         & = \hbar\Omega                                                                            \\
    \mel{J\Omega \pm 1}{\sop{J}_\mp}{J\Omega} & = \hbar\exp[i(\phi_\Omega - \phi_{\Omega \mp 1})]\sqrt{J(J + 1) - \Omega(\Omega \pm 1)}.
\end{align*}
Total electron orbital angular momentum $\vop{L}$:
\begin{align*}
    \mel{L\Lambda}{\vop{L}^2}{L\Lambda}         & = \hbar^2L(L + 1)                                                                            \\
    \mel{L\Lambda}{\sop{L}_z}{L\Lambda}         & = \hbar\Lambda                                                                               \\
    \mel{L\Lambda \pm 1}{\sop{L}_\pm}{L\Lambda} & = \hbar\exp[i(\phi_\Lambda - \phi_{\Lambda \pm 1})]\sqrt{L(L + 1) - \Lambda(\Lambda \pm 1)}.
\end{align*}
Total electron spin angular momentum $\vop{S}$:
\begin{align*}
    \mel{S\Sigma}{\vop{S}^2}{S\Sigma}         & = \hbar^2S(S + 1)                                                                        \\
    \mel{S\Sigma}{\sop{S}_z}{S\Sigma}         & = \hbar\Sigma                                                                            \\
    \mel{S\Sigma \pm 1}{\sop{S}_\pm}{S\Sigma} & = \hbar\exp[i(\phi_\Sigma - \phi_{\Sigma \pm 1})]\sqrt{S(S + 1) - \Sigma(\Sigma \pm 1)}.
\end{align*}
Following the Condon and Shortley phase convention \cite[48]{condonTheoryAtomicSpectra1935}, we require the $\sop{A}_x$ matrix elements for both normal and anomalous $\vop{A}$ to be both real and positive, thereby requiring the $\phi_\alpha$ phase factors for all $2A + 1$ basis functions to be equal.
Applying this convention simplifies the matrix elements for the ladder operators to
\begin{align*}
    \mel{J\Omega \pm 1}{\sop{J}_\mp}{J\Omega}   & = \hbar\sqrt{J(J + 1) - \Omega(\Omega \pm 1)}   \\
    \mel{L\Lambda \pm 1}{\sop{L}_\pm}{L\Lambda} & = \hbar\sqrt{L(L + 1) - \Lambda(\Lambda \pm 1)} \\
    \mel{S\Sigma \pm 1}{\sop{S}_\pm}{S\Sigma}   & = \hbar\sqrt{S(S + 1) - \Sigma(\Sigma \pm 1)}.
\end{align*}

\section{The Molecular Hamiltonian}

The molecular Hamiltonian is \cite[341]{brownRotationalSpectroscopyDiatomic2003}
\begin{equation*}
    \sop{H} = \sop{H}_{ev} + \sop{H}_r + \sop{H}_{so} + \sop{H}_{ss} + \sop{H}_{sr} + \sop{H}_{ld}.
\end{equation*}
When possible, IUPAC recommendations \cite{hirotaSymbolsFineHyperfine1994} are followed.

\subsection{Rotation Operator}

For all diatomic molecules, the rotation operator is
\begin{equation*}
    \sop{H}_r = B\vop{N}^2 - D\vop{N}^4 + H\vop{N}^6 + L\vop{N}^8 + M\vop{N}^{10} + P\vop{N}^{12} + \dotsb.
\end{equation*}

\subsection{Spin-Orbit Operator}

For $\Lambda > 0$ states with $S > 0$, the spin-orbit operator is
\begin{equation*}
    \sop{H}_{so} = A\ab(\vop{L}\vdot\vop{S}) + \frac{A_D}{2}\acomm*{\vop{N}^2}{\vop{L}\vdot\vop{S}} + \frac{A_H}{2}\acomm*{\vop{N}^4}{\vop{L}\vdot\vop{S}} + \frac{A_L}{2}\acomm*{\vop{N}^6}{\vop{L}\vdot\vop{S}} + \frac{A_M}{2}\acomm*{\vop{N}^8}{\vop{L}\vdot\vop{S}}.
\end{equation*}
For $\Lambda > 0$ states with $S > 1$, the additional term
\begin{equation*}
    \eta\sop{L}_z\sop{S}_z\ab[\sop{S}_z^2 - \frac{1}{5}\ab(3\vop{S}^2 - 1)]
\end{equation*}
is required \cite[140]{brownHigherOrderFineStructure1981}.

\subsection{Spin-Spin Operator}

For states with $S > 1/2$, the spin-spin operator is
\begin{equation*}
    \sop{H}_{ss} = \frac{2}{3}\lambda\ab(3\sop{S}_z^2 - \vop{S}^2) + \frac{1}{2}\lambda_D\acomm*{\frac{2}{3}\ab(3\sop{S}_z^2 - \vop{S}^2)}{\vop{N}^2} + \frac{1}{2}\lambda_H\acomm*{\frac{2}{3}\ab(3\sop{S}_z^2 - \vop{S}^2)}{\vop{N}^4}.
\end{equation*}
When $S > 3/2$, the term
\begin{equation*}
    \frac{1}{12}\theta\ab(35\sop{S}_z^4 - 30\vop{S}^2\sop{S}_z^2 + 25\sop{S}_z^2 - 6\vop{S}^2 + 3\vop{S}^4)
\end{equation*}
must also be included \cite[472]{brownLTypeDoublingParameters1987}.

\subsection{Spin-Rotation Operator}

When $S > 0$, the spin-rotation operator is
\begin{equation*}
    \sop{H}_{sr} = \gamma\ab(\vop{N}\vdot\vop{S}) + \frac{1}{2}\gamma_D\acomm*{\vop{N}\vdot\vop{S}}{\vop{N}^2} + \frac{1}{2}\gamma_H\acomm*{\vop{N}\vdot\vop{S}}{\vop{N}^4} + \frac{1}{2}\gamma_L\acomm*{\vop{N}\vdot\vop{S}}{\vop{N}^6}.
\end{equation*}
For states with $S > 1$, the additional term
\begin{equation*}
    -\sqrt{70/3}\gamma_ST_0^2\acomm*{T^1(\vop{J})}{T^3(\vop{S})}
\end{equation*}
is required \cite[320]{cheungFourierTransformSpectroscopy1984}.

\subsection{\texorpdfstring{$\Lambda$}{Λ} Doubling Operator for \texorpdfstring{$\Pi$}{Π} States}

For any $\Pi$ state, the $\Lambda$ doubling operator is \cite[488]{brownLambdaTypeDoublingParameters1979}
\begin{equation*}
    \sop{H}_{ld} = \frac{1}{2}q\ab(\sop{N}_+^2e^{-2i\phi} + \sop{N}_-^2e^{2i\phi}) + \frac{1}{4}\acomm*{\sop{N}_+^2e^{-2i\phi} + \sop{N}_-^2e^{2i\phi}}{q_D\vop{N}^2 + q_H\vop{N}^4 + q_L\vop{N}^6}.
\end{equation*}
For $\Pi$ states with $S > 0$, the term
\begin{equation*}
    -\frac{1}{2}p\ab(\sop{N}_+\sop{S}_+e^{-2i\phi} + \sop{N}_-\sop{S}_-e^{2i\phi}) - \frac{1}{4}\acomm*{\sop{N}_+\sop{S}_+e^{-2i\phi} + \sop{N}_-\sop{S}_-e^{2i\phi}}{p_D\vop{N}^2 + p_H\vop{N}^4 + p_L\vop{N}^6}
\end{equation*}
is added. When $S > 1/2$, the extra term
\begin{equation*}
    \frac{1}{2}o\ab(\sop{S}_+^2e^{-2i\phi} + \sop{S}_-^2e^{2i\phi}) + \frac{1}{4}\acomm*{\sop{S}_+^2e^{-2i\phi} + \sop{S}_-^2e^{2i\phi}}{o_D\vop{N}^2 + o_H\vop{N}^4 + o_L\vop{N}^6}
\end{equation*}
is also required.

\begin{table}[H]
    \centering
    \renewcommand{\arraystretch}{1.5}
    \caption{First-order terms in the diatomic Hamiltonian \cite[232-233]{westernPGOPHERProgramSimulating2017}.}
    \begin{tabular}{lll}
        \toprule
        \textbf{Operator} & \textbf{Equation}                                                                                         & \textbf{Condition}      \\
        \midrule
        Rotation          & $B\vop{N}^2$                                                                                              & always                  \\
        \cmidrule{2-3}
        \multirow{2}{*}{Spin-Rotation}
        & $\gamma\ab(\vop{N}\vdot\vop{S})$                                                                          & $S > 0$                 \\
        & $-\sqrt{70/3}\gamma_ST_0^2\acomm*{T^1(\vop{J})}{T^3(\vop{S})}$                                            & $S > 1$                 \\
        \cmidrule{2-3}
        \multirow{2}{*}{Spin-Orbit}
        & $A\ab(\vop{L}\vdot\vop{S})$                                                                               & $\Lambda > 0$, $S > 0$  \\
        & $\eta\sop{L}_z\sop{S}_z\ab[\sop{S}_z^2 - \frac{1}{5}\ab(3\vop{S}^2 - 1)]$                                 & $\Lambda > 0$, $S > 1$  \\
        \cmidrule{2-3}
        \multirow{2}{*}{Spin-Spin}
        & $\frac{2}{3}\lambda\ab(3\sop{S}_z^2 - \vop{S}^2)$                                                         & $S > 1/2$               \\
        & $\frac{1}{12}\theta\ab(35\sop{S}_z^4 - 30\vop{S}^2\sop{S}_z^2 + 25\sop{S}_z^2 - 6\vop{S}^2 + 3\vop{S}^4)$ & $S > 3/2$               \\
        \cmidrule{2-3}
        \multirow{3}{*}{$\Lambda$ Doubling}
        & $\frac{1}{2}q\ab(\sop{N}_+^2e^{-2i\phi} + \sop{N}_-^2e^{2i\phi})$                                         & $\Pi$ states            \\
        & $-\frac{1}{2}p\ab(\sop{N}_+\sop{S}_+e^{-2i\phi} + \sop{N}_-\sop{S}_-e^{2i\phi})$                          & $\Pi$ states, $S > 0$   \\
        & $\frac{1}{2}o\ab(\sop{S}_+^2e^{-2i\phi} + \sop{S}_-^2e^{2i\phi})$                                         & $\Pi$ states, $S > 1/2$ \\
        \bottomrule
    \end{tabular}
\end{table}

\section{Matrix Form of Operators}

The rotational operator $\vop{N}^2$ can be expressed as a matrix for any given state. For a $\state{2}{\Sigma}$ state, the first two matrices are
\begin{equation*}
    \vop{N}^2 =
    \begin{bmatrix}
        x + \frac{1}{4}         & -\sqrt{x + \frac{1}{4}} \\
        -\sqrt{x + \frac{1}{4}} & x + \frac{1}{4}
    \end{bmatrix}
\end{equation*}
and
\begin{equation*}
    \vop{N}^4 = \vop{N}^2\vop{N}^2 =
    \begin{bmatrix}
        \ab(x + \frac{1}{4})^2 + x + \frac{1}{4} & -2\ab(x + \frac{1}{4})^{3/2}             \\
        -2\ab(x + \frac{1}{4})^{3/2}             & \ab(x + \frac{1}{4})^2 + x + \frac{1}{4}
    \end{bmatrix}.
\end{equation*}
Higher powers can be formed as such:
\begin{equation*}
    \vop{N}^{2k} = \vop{N}^{2k - 1}\vop{N}^2.
\end{equation*}

\subsection{Matrix Elements of Linear Operators}

The matrix elements of any general linear operator $\vop{\Omega}$ can be written as
\begin{equation*}
    \mel{i}{\vop{\Omega}}{j} = \vop{\Omega}_{ij}.
\end{equation*}
For products of operators, the matrix elements are \cite[25]{shankarPrinciplesQuantumMechanics1994}
\begin{equation*}
    \mel{i}{\vop{\Omega}\vop{\Lambda}}{j} = \mel{i}{\vop{\Omega}I\vop{\Lambda}}{j} = \sum_k\mel{i}{\vop{\Omega}}{k}\mel{k}{\vop{\Lambda}}{j} = \sum_k\vop{\Omega}_{ik}\vop{\Lambda}_{kj}.
\end{equation*}
A more complex case can now be evaluated:
\begin{align*}
    \mel{i}{\acomm{\vop{N}\vdot\vop{S}}{\vop{N}^2}}{j} & = \mel{i}{(\vop{N}\vdot\vop{S})\vop{N}^2}{j} + \mel{i}{\vop{N}^2(\vop{N}\vdot\vop{S})}{j}                                 \\
    & = \sum_k\mel{i}{\vop{N}\vdot\vop{S}}{k}\mel{k}{\vop{N}^2}{j} + \sum_k\mel{i}{\vop{N}^2}{k}\mel{k}{\vop{N}\vdot\vop{S}}{j} \\
    & = \sum_k(\vop{N}\vdot\vop{S})_{ik}\vop{N}^2_{kj} + \sum_k\vop{N}^2_{ik}(\vop{N}\vdot\vop{S})_{kj}
\end{align*}
