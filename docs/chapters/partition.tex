\chapter{Partition Functions}

\section{Partition Functions}

\subsection{General Form}

General form of the Boltzmann population distribution \cite[8]{hansonSpectroscopyOpticalDiagnostics2016}
\begin{equation*}
    \frac{N_i}{N} = \frac{g_i}{Q}\exp\ab(-\frac{\varepsilon_i}{kT})
\end{equation*}
General form of the partition function \cite[8]{hansonSpectroscopyOpticalDiagnostics2016}
\begin{equation*}
    Q = \sum_i g_i\exp\ab(-\frac{\varepsilon_i}{kT})
\end{equation*}
The total molecular partition function can be expressed as the product of the individual translational, electronic, vibrational, rotational, and nuclear components for symmetric molecules is \cite[80]{hansonSpectroscopyOpticalDiagnostics2016}
\begin{equation*}
    Q = Q_tQ_eQ_vQ_rQ_n.
\end{equation*}

\subsection{Nuclear}

See \cite[81]{hansonSpectroscopyOpticalDiagnostics2016}
\begin{equation*}
    Q_n = \prod_{i = 1}^L (2I_n + 1)
\end{equation*}

\subsection{Rotational}
Because of the rotational symmetry of certain molecules, a symmetry parameter $\sigma$ must be added to the rotational partition function to account for the multiple molecular orientations that are indistinguishable.
See \cite[125]{herzbergMolecularSpectraMolecular1950}
\begin{equation*}
    Q_r = \frac{1}{\sigma} \sum g_r\exp\ab(-\frac{\varepsilon_r}{kT})
\end{equation*}
The degeneracy of each state is
\begin{equation*}
    g_r = 2J + 1
\end{equation*}
and the rotational energy can be expressed as the term value
\begin{equation*}
    \varepsilon_r = F(J)hc.
\end{equation*}
Therefore, the rotational partition function is
\begin{equation*}
    Q_r = \frac{1}{\sigma} \sum (2J + 1)\exp\ab(-\frac{F(J)hc}{kT}).
\end{equation*}
For sufficiently large $T$ or small $B$, the rotational partition function for a molecule in the high-temperature limit is \cite[17]{hansonSpectroscopyOpticalDiagnostics2016} and \cite[125]{herzbergMolecularSpectraMolecular1950}
\begin{equation*}
    Q_r \approx \frac{1}{\sigma} \int_0^{\infty} (2J + 1)\exp\ab(-\frac{BJ(J + 1)hc}{kT}) \odif{J} = \frac{1}{\sigma}\frac{kT}{hcB}
\end{equation*}
The Boltzmann fraction is then
\begin{equation*}
    \frac{N_J}{N} = \frac{(2J + 1)}{Q_r}\exp\ab(-\frac{F(J)hc}{kT})
\end{equation*}

\subsection{Effective Rotational}

The effective rotational partition function for symmetric molecules is \cite[80]{hansonSpectroscopyOpticalDiagnostics2016}
\begin{equation*}
    Q'_r = Q_rQ_n
\end{equation*}
Similarly, the effective rotational degeneracy is \cite[80]{hansonSpectroscopyOpticalDiagnostics2016}
\begin{equation*}
    g'_r = g_rg_n
\end{equation*}
For homonuclear diatomic molecules, the effective rotational degeneracy is \cite[84]{hansonSpectroscopyOpticalDiagnostics2016}
\begin{equation*}
    g'_r = \frac{(2J + 1)}{2}\ab[(2I + 1)^2 \pm (2I + 1)]
\end{equation*}

\subsection{Vibrational}

See \cite[123]{herzbergMolecularSpectraMolecular1950}
\begin{equation*}
    Q_v = \sum g_v\exp\ab(-\frac{\varepsilon_v}{kT})
\end{equation*}
The vibrational degeneracy of each state is
\begin{equation*}
    g_v = 1,
\end{equation*}
and the vibrational energy can be expressed as the term value
\begin{equation*}
    \varepsilon_v = G(v)hc.
\end{equation*}
Therefore, the vibrational partition function is
\begin{equation*}
    Q_v = \sum \exp\ab(-\frac{G(v)hc}{kT}).
\end{equation*}
The Boltzmann fraction is then
\begin{equation*}
    \frac{N_v}{N} = \frac{1}{Q_v}\exp\ab(-\frac{G(v)hc}{kT})
\end{equation*}
The vibrational partition function can also be written in terms of the zero-point vibrational energy as
\begin{equation*}
    Q_v = \sum \exp\ab(-\frac{[G(v) - G(0)]hc}{kT}).
\end{equation*}
The Boltzmann fraction is therefore
\begin{equation*}
    \frac{N_v}{N} = \frac{1}{Q_v}\exp\ab(-\frac{[G(v) - G(0)]hc}{kT}).
\end{equation*}

\subsection{Electronic}

See \cite[544]{andersonHypersonicHighTemperatureGas2019}
\begin{equation*}
    Q_e = \sum g_e\exp\ab(-\frac{\varepsilon}{kT})
\end{equation*}
The electronic degeneracy depends on the individual states. The electronic energy can be expressed as the term value
\begin{equation*}
    \varepsilon_e = T_ehc.
\end{equation*}
Therefore, the electronic partition function is
\begin{equation*}
    Q_e = \sum g_e\exp\ab(-\frac{T_ehc}{kT}).
\end{equation*}
The Boltzmann fraction is therefore
\begin{equation*}
    \frac{N_e}{N} = \frac{g_e}{Q_e}\exp\ab(-\frac{T_ehc}{kT}).
\end{equation*}

\subsection{Translational}

See \cite[544]{andersonHypersonicHighTemperatureGas2019}
\begin{equation*}
    Q_t = \ab(\frac{2\pi mkT}{h^2})^{3/2}V,
\end{equation*}
where $V$ is the volume of the system.
